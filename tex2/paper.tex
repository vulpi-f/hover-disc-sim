% !TEX program = pdflatex
\documentclass[11pt,a4paper]{article}

\usepackage[utf8]{inputenc}
\usepackage[T1]{fontenc}
\usepackage{lmodern}
\usepackage{geometry}
\geometry{margin=1in}
\usepackage{amsmath,amssymb}
\usepackage{siunitx}
\usepackage{graphicx}
\usepackage[font=small,labelfont=bf,labelsep=endash]{caption}
\usepackage{xcolor}
\usepackage{booktabs}
\usepackage{cite}
\usepackage{hyperref}
\usepackage{float}
\usepackage{listings}
\lstdefinestyle{codeblock}{
  basicstyle=\ttfamily\small,
  columns=fullflexible,
  keepspaces=true,
  breaklines=true,
  breakatwhitespace=true,
  postbreak=\mbox{\textcolor{gray}{$\hookrightarrow$}\space},
  frame=single,
  framerule=0.2pt,
  xleftmargin=1em, xrightmargin=0em,
  aboveskip=0.8\baselineskip,
  belowskip=0.8\baselineskip,
  captionpos=t
}

\hypersetup{colorlinks=true,linkcolor=blue,citecolor=blue,urlcolor=blue}

\title{A Momentum-Based Modeling of Jet-Induced Sealing Flow}
\author{[Author Name]}
\date{}

\begin{document}
\maketitle

\begin{abstract}
We present a steady, incompressible and axisymmetric model for jet-induced sealing in thin gaps.
The formulation is derived directly from local and global conservation of mass and momentum in three coupled regions: the confined core, the vertical jet, and the radial wall-jet after impact.
No interface tuning is introduced; the sealing pressure, jet deflection, leakage and lift emerge from the balance equations closed with measurable correlations (e.g.\ wall shear in the wall-jet).
We provide a rigorous description of the system geometry, operating principle and objectives, the governing equations and interface conditions, and a complete description of the numerical solver with pseudocode.
\end{abstract}

\section{Introduction}
High-speed jets impinging on confined gaps can generate pressure distributions capable of sustaining external loads and reducing leakage.
This mechanism, here termed \emph{jet-induced sealing}, originates from the redistribution of momentum between the incoming jet, the confined core flow, and the wall-jet along the adjacent surface.
A physically consistent description must \emph{derive} sealing pressures and leakage from the dynamic equilibrium of the flow.
\begin{figure}[H]
  \centering
  \includegraphics[width=0.95\linewidth]{../figs/schema_geometry.png}
  \caption{Schematic of the hovering disc with two concentric jets: the outer annular curtain and the central make-up flow.}
  \label{fig:geometry}
\end{figure}

\section{System Description and Objectives}
\label{sec:system}
\paragraph{Geometry and flow layout.}
We consider two horizontal surfaces separated by a nominal gap $h$ with $h\ll R$, where $R$ is the characteristic radial extent.
An annular slot of width $b_0$ at radius $R^{-}$ injects a vertical jet of density $\rho$ and mean speed $U_j$ directed towards the base surface.
Upon impingement, the jet turns into a radial wall-jet of thickness $\delta(r)$ along the base.
The confined \emph{core} occupies $0\le r\le R^{-}$ within the gap.

\paragraph{Operating principle.}
The vertical momentum of the jet is partially converted into pressure in a narrow turning region near $r\approx r_t\lesssim R^{-}$, creating a pedestal overpressure $p_{\mathrm{ped}}(r)$ which raises the core pressure $p_c$ and reduces leakage.
The resulting pressure field produces a lift $L=\int_0^{R^{-}}(p_c-p_0)\,2\pi r\,dr$ that can support an external load $W$ while keeping leakage $\dot m_{\mathrm{leak}}$ acceptably low.

\paragraph{Objectives.}
Given $(h,R,b_0)$, fluid properties $(\rho,\mu)$ and a load $W$, we seek the unknowns
\begin{align*}
p_c(r),\;\bar u(r),\;U_j\ \text{(or }\dot m_j),\;U_w(r),\;\delta(r),\;p_{\mathrm{ped}}(r),\;r_t,\;\dot m_{\mathrm{leak}},
\end{align*}
together with performance metrics (lift--height map $L(h)$, required jet power $\dot W_{\mathrm{jet}}$, sealing and leakage indices, and efficiency $\eta$).

\section{Physical Model}
\label{sec:model}
The flow is modeled as steady, axisymmetric and incompressible (air at low Mach).
Three coupled regions are considered:
(i) the \emph{core} thin film in $0\le r\le R^{-}$;
(ii) the \emph{vertical jet} issuing from the slot and impinging on the base;
(iii) the \emph{radial wall-jet} after turning.
Coupling follows from continuity of mass and momentum.

\subsection{Core (thin-film / lubrication)}
Depth-averaging Navier--Stokes in the limit $h\ll R$ yields
\begin{equation}
\bar u(r) = -\frac{h^2}{12\mu}\frac{dp_c}{dr},\qquad
\frac{1}{r}\frac{d}{dr}\!\left(r\,\rho\,h\,\bar u\right)=s(r),
\label{eq:lube_mass}
\end{equation}
where $s(r)$ accounts for any distributed sources (typically $s=0$ away from the jet).
Eliminating $\bar u$,
\begin{equation}
\frac{1}{r}\frac{d}{dr}\!\left(-\,r\,\frac{\rho h^3}{12\mu}\frac{dp_c}{dr}\right)=s(r).
\label{eq:lube_pressure}
\end{equation}
Boundary conditions: symmetry at $r=0$ ($dp_c/dr=0$) and an \emph{unknown} edge pressure $p_c(R^-)$ provided by the turning region.
The lift must balance the applied load:
\begin{equation}
L=\int_0^{R^-}(p_c-p_0)\,2\pi r\,dr = W.
\label{eq:lift}
\end{equation}

\subsection{Jet and turning region}
Let $\dot m_j=\rho U_j b_0$ be the jet mass flow per unit circumference.
A vertical momentum balance over a control volume from the slot to the base gives
\begin{equation}
\dot m_j U_j = \int_{r_t^-}^{r_t^+}\!\left[p_{\mathrm{ped}}(r)-p_0\right]\,dr + \Delta{\mathcal M}_z,
\label{eq:jet_momentum}
\end{equation}
where $\Delta{\mathcal M}_z$ accounts for viscous loss and three-dimensional turning; it is not prescribed but determined implicitly by consistency with the wall-jet momentum (next subsection).
The initiation of the wall-jet at $r_t$ obeys flux continuity:
\begin{equation}
q(r_t)=\frac{\dot m_j}{2\pi r_t\rho}, \qquad
m(r_t)=\frac{\dot m_j U_j}{2\pi r_t}.
\label{eq:turn_init}
\end{equation}

\subsection{Radial wall-jet}
An integral formulation for the wall-jet of thickness $\delta(r)$ and characteristic speed $U_w(r)$ is adopted.
Define $q\equiv\int_0^\delta u_r\,dz\simeq U_w\delta$ and $m\equiv\int_0^\delta \rho u_r^2\,dz \simeq \rho U_w^2\delta$.
Then
\begin{align}
\frac{d}{dr}(\rho q) &= 2\pi r\,\rho\,E\,U_w, \label{eq:wj_mass}\\
\frac{d}{dr}(\rho m) &= -\,\tau_w\,2\pi r - \frac{dp}{dr}\,2\pi r\,\delta, \label{eq:wj_mom}
\end{align}
where $\tau_w=\tfrac12\rho C_f U_w^2$ and $E\ge 0$ admits entrainment from the surrounding fluid when relevant.
Close to the wall-jet inception, $p\approx p_0$ except within the narrow turning region where $p=p_{\mathrm{ped}}$.
Standard smooth-wall correlations provide $C_f=C_f(\mathrm{Re}_\delta)$ and a growth law $d\delta/dr$; the model is insensitive to the particular choice as long as it is physically consistent and measured.

\subsection{Interface and global mass balance}
The edge pressure of the core equals the local pressure at the turning rim:
\begin{equation}
p_c(R^-)=p_0 + p_{\mathrm{ped}}(R^-).
\label{eq:edge_pressure}
\end{equation}
The leakage through the core rim follows from \eqref{eq:lube_mass}:
\begin{equation}
\dot m_{\mathrm{leak}} = 2\pi R^-\,\rho\,h\,\bar u(R^-) = -\,\frac{\pi \rho h^3}{6\mu}\,R^-\,\left.\frac{dp_c}{dr}\right|_{R^-}\!.
\label{eq:leak}
\end{equation}
Global mass conservation relates inlet, jet and leakage:
\begin{equation}
\dot m_{\mathrm{in}} = \dot m_j + \dot m_{\mathrm{leak}}.
\label{eq:global_mass}
\end{equation}

\section{Non-dimensionalization (for analysis and scaling)}
Let $r^\ast=r/R^-$, $p^\ast=(p-p_0)/(\rho U_j^2)$, $u^\ast=\bar u/U_j$, $\delta^\ast=\delta/b_0$ and define $\lambda=h/b_0\ll 1$ and $\mathrm{Re}_j=U_j b_0/\nu$.
Equations \eqref{eq:lube_pressure}--\eqref{eq:global_mass} reduce to a compact set revealing the dominance of thin-film resistance and the wall-jet friction in setting $p_c(R^-)$; the seal and leakage indices follow as
\begin{equation}
\Pi_{\mathrm{seal}}=\frac{\rho U_j^2 b_0}{h\,\overline{p_c}},\qquad
\Pi_{\mathrm{leak}}=\frac{\dot m_{\mathrm{leak}}}{\rho U_j\,2\pi R^- b_0},
\end{equation}
useful for design comparisons at fixed geometry.

\section{Numerical Implementation}
\label{sec:numerics}
We solve the coupled problem by iterating between the core (thin-film ODE), the turning region momentum closure and the wall-jet integral equations until mass and lift equilibria are simultaneously satisfied.
All steps are steady; no time marching is required.

\paragraph{Discretization.}
The core radius $[0,R^-]$ is discretized on a collocated grid $r_i$ ($i=0,\dots,N$) refined near $R^-$.
Equation \eqref{eq:lube_pressure} is integrated in conservative finite-volume form with second-order central differences for $dp_c/dr$ and boundary conditions $dp_c/dr|_{r_0}=0$, $p_c(r_N)=p_{\mathrm{edge}}$.
The wall-jet ODE system \eqref{eq:wj_mass}--\eqref{eq:wj_mom} plus a growth law for $\delta$ is advanced from $r_t$ to an outer radius $R_{\max}$ using an explicit strong-stability-preserving Runge--Kutta scheme; the initial conditions at $r_t$ are given by \eqref{eq:turn_init}.

\paragraph{Coupling and unknowns.}
Primary unknowns are $\{p_c(r_i)\}$, $p_{\mathrm{edge}}$, $\dot m_j$ (or $U_j$), and the wall-jet fields $\{q(r),m(r),\delta(r)\}$ for $r\ge r_t$.
At each nonlinear iteration we enforce: (i) edge pressure consistency \eqref{eq:edge_pressure}, (ii) global mass \eqref{eq:global_mass}, and (iii) lift balance \eqref{eq:lift}.

\paragraph{Convergence criteria.}
We require the residuals
\begin{align*}
\mathcal R_p &= p_{\mathrm{edge}}-(p_0+p_{\mathrm{ped}}(R^-)),\\
\mathcal R_m &= \dot m_{\mathrm{in}}-(\dot m_j+\dot m_{\mathrm{leak}}),\\
\mathcal R_L &= L-W
\end{align*}
to satisfy $|\mathcal R_\bullet| \le \varepsilon_\bullet$ with tolerances $\varepsilon_p,\varepsilon_m,\varepsilon_L$ typically set to $10^{-5}$ in normalized units.

\paragraph{Algorithm (pseudocode).}
The following high-level pseudocode outlines the complete solver.
\begin{lstlisting}[style=codeblock,language={} ,caption={Nonlinear solver for the coupled jet-induced sealing model.}]
Given geometry (R_minus, b0, h), fluid (rho, mu), load W, inlet m_in
Choose grid r[0..N] on [0, R_minus] (refined near R_minus)
Initialize guesses: Uj (or m_j), p_edge, rt ~ R_minus, delta(rt), Uw(rt)
Set tolerances eps_p, eps_m, eps_L and iteration limits

repeat (outer nonlinear loop)
  # 1) Wall-jet integration (requires jet init and pedestal pressure)
  compute m_j = rho * Uj * b0
  set q(rt) = m_j / (2*pi*rt*rho)
  set m(rt) = m_j * Uj / (2*pi*rt)
  integrate wall-jet ODEs for {q(r), m(r), delta(r)} from r=rt to R_max
    using correlations Cf(Re_delta) and growth d(delta)/dr
  from jet vertical momentum, infer pedestal pressure profile p_ped(r)
    such that momentum turning is satisfied and matches wall-jet m(r)
  evaluate p_edge_candidate = p0 + p_ped(R_minus)

  # 2) Core thin-film solve with edge pressure
  solve lubrication equation for p_c(r) on [0, R_minus] with:
    dp_c/dr|_{r=0}=0 and p_c(R_minus)=p_edge_candidate
  compute leakage: m_leak = 2*pi*R_minus*rho*h*ubar(R_minus)
    where ubar = -(h^2/(12*mu))*dp_c/dr

  # 3) Global constraints and updates
  residuals:
    R_p = p_edge - p_edge_candidate
    R_m = m_in - (m_j + m_leak)
    R_L = (integral_0^{R-} (p_c-p0) 2*pi*r dr) - W

  if max(|R_p|, |R_m|, |R_L|) < tolerances -> CONVERGED

  # 4) Nonlinear update (quasi-Newton / fixed-point mix)
  update (Uj, p_edge, rt) using a 3x3 Jacobian-free secant step
    e.g., Broyden update with line search to reduce residual norm
  optionally under-relax delta- and Cf-related profiles if needed

until converged or iteration limit reached
post-process: fields {p_c, ubar, q, m, delta}, metrics {L(h), Wdot_jet, Pi_seal, Pi_leak, eta}
\end{lstlisting}

\paragraph{Complexity and robustness.}
The core solve is a banded linear system $\mathcal{O}(N)$.
The wall-jet step is $\mathcal{O}(M)$ with $M$ radial steps from $r_t$ to $R_{\max}$.
The outer Broyden loop typically converges in ${\sim}5$--$15$ iterations for practical tolerances.
Under-relaxation of $(U_j,p_{\mathrm{edge}},r_t)$ and slope limiting on $\delta(r)$ improve robustness at high Reynolds numbers.

\section{Governing Equations (compact summary)}
For quick reference, the model equations are:
\begin{align*}
&\text{Core:}&
\bar u &= -\frac{h^2}{12\mu}\frac{dp_c}{dr}, \quad
\frac{1}{r}\frac{d}{dr}\!\left(r\,\rho h \bar u\right)=s(r),\quad
L=\int_0^{R^-}(p_c-p_0)2\pi r\,dr=W.\\[3pt]
&\text{Jet:}&
\dot m_j &= \rho U_j b_0,\quad
\dot m_j U_j = \int_{r_t^-}^{r_t^+}(p_{\mathrm{ped}}-p_0)\,dr + \Delta\mathcal M_z.\\[3pt]
&\text{Wall-jet:}&
\frac{d}{dr}(\rho q) &= 2\pi r\rho E U_w,\quad
\frac{d}{dr}(\rho m) = -\tau_w 2\pi r - \frac{dp}{dr}2\pi r\delta,\quad
\tau_w=\tfrac12\rho C_f U_w^2.\\[3pt]
&\text{Coupling:}&
p_c(R^-)&=p_0+p_{\mathrm{ped}}(R^-),\quad
\dot m_{\mathrm{leak}} = 2\pi R^- \rho h \bar u(R^-),\quad
\dot m_{\mathrm{in}}=\dot m_j+\dot m_{\mathrm{leak}}.
\end{align*}

\section{Simulation Outputs}
\label{sec:simulation-outputs}
The results produced from the model are shown in this section.
Unless otherwise noted, fields are reported in non-dimensional form with the jet speed and slot width as reference scales.

\begin{figure}[H]
  \centering
  \includegraphics[width=0.95\linewidth]{../figs/quiver_velocity.png}
  \caption{Non-dimensional velocity field (quiver).
Vectors show $(\hat u,\,S\hat w)$ for isotropic visual scaling; the pattern reflects the pedestal pressure induced by the turning jet.}
  \label{fig:quiver}
\end{figure}

\begin{figure}[H]
  \centering
  \includegraphics[width=0.95\linewidth]{../figs/cmap_speed.png}
  \caption{Colormap of the non-dimensional isotropic speed magnitude $\hat V_{\mathrm{iso}}=\sqrt{\hat u^{\,2}+S^{2}\hat w^{\,2}}$.}
  \label{fig:cmap_speed}
\end{figure}

\begin{figure}[H]
  \centering
  \includegraphics[width=0.95\linewidth]{../figs/cmap_ur.png}
  \caption{Colormap of the non-dimensional radial component magnitude $|\hat u|$.}
  \label{fig:cmap_ur}
\end{figure}

\begin{figure}[H]
  \centering
  \includegraphics[width=0.95\linewidth]{../figs/cmap_uz.png}
  \caption{Colormap of the non-dimensional axial component magnitude $|\hat w|$.}
  \label{fig:cmap_uz}
\end{figure}

\begin{figure}[H]
  \centering
  \includegraphics[width=0.95\linewidth]{../figs/cmap_pressure.png}
  \caption{Colormap of the non-dimensional pressure $\hat p$.}
  \label{fig:cmap_p}
\end{figure}

\paragraph{Performance and stability analysis.}
From the converged solutions the lift--height characteristic $L(h)$ is evaluated to assess static stability.
A stable hovering point satisfies $dL/dh < 0$ near the nominal gap.
The required jet power is estimated as
\begin{equation}
  \dot W_{\mathrm{jet}} = \frac{1}{2}\dot m_{\mathrm{out}} U_{\mathrm{out}}^2 /(1+K_{\mathrm{turn}}),
\end{equation}
and the non-dimensional sealing and leakage indices are
\begin{equation}
  \Pi_{\mathrm{seal}} = \frac{\rho_j U_{\mathrm{out}}^2 b_0}{h\,\overline{p_c}},\qquad
  \Pi_{\mathrm{leak}} = \frac{\dot m_{\mathrm{leak}}}{\rho U_{\mathrm{out}} 2\pi R^{-} b_0}.
\end{equation}
Together with the seal number $\Sigma$ and the efficiency $\eta=\dfrac{W h}{\dot W_{\mathrm{jet}}}$, these provide compact performance metrics for design optimization.

\section{Discussion}
This momentum-based framework determines the sealing pressure and lift directly from the interplay between the impinging jet and the confined core flow.
The approach ensures that the sealing effect arises as a natural consequence of conservation laws, with only standard, measurable correlations in the wall-jet closure.

\section{Conclusions}
We have formulated a self-consistent, momentum-driven description of jet-induced sealing flows.
The model couples lubrication pressure in the core with integral balances in the jet and wall-jet regions, and provides an implementable numerical algorithm with clear convergence criteria.
All relevant physical quantities emerge from conservation principles, offering a transparent basis for prediction and design.

\bibliographystyle{plain}
\bibliography{refs}

\end{document}
