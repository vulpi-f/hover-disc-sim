% !TEX program = pdflatex
\documentclass[11pt,a4paper]{article}

\usepackage[utf8]{inputenc}
\usepackage[T1]{fontenc}
\usepackage{lmodern}
\usepackage{geometry}
\geometry{margin=1in}
\usepackage{amsmath,amssymb}
\usepackage{siunitx}
\usepackage{graphicx}
\usepackage[font=small,labelfont=bf,labelsep=endash]{caption}
\usepackage{xcolor}
\usepackage{booktabs}
\usepackage{cite}
\usepackage{hyperref}
\usepackage{float}
\usepackage{listings}
\lstdefinestyle{codeblock}{
  basicstyle=\ttfamily\small,
  columns=fullflexible,
  keepspaces=true,
  breaklines=true,
  breakatwhitespace=true,
  postbreak=\mbox{\textcolor{gray}{$\hookrightarrow$}\space},
  frame=single,
  framerule=0.2pt,
  xleftmargin=1em, xrightmargin=0em,
  aboveskip=0.8\baselineskip,
  belowskip=0.8\baselineskip,
  captionpos=t
}

\hypersetup{colorlinks=true,linkcolor=blue,citecolor=blue,urlcolor=blue}

\title{Aerodynamic Levitator with Annular Jet Sealing:\\
Concept, Geometry and Operating Principle}
\author{[Author Name]}
\date{}

\begin{document}
\maketitle

\begin{abstract}
We present the conceptual design of an \emph{aerodynamic levitation system} based on a circular disc that produces a sealing \emph{air curtain} along its outer rim. 
Unlike hovercrafts relying on Coandă-effect skirts or drones generating lift from rotor thrust, this device uses an annular high-speed jet to confine a nearly static, high-pressure air cushion trapped between the disc and the ground. 
The pressurized region sustains an external load while the curtain jet minimizes leakage to the surroundings. 
A central air inlet compensates the residual losses, maintaining the cushion pressure in steady equilibrium.
This paper describes the geometry, operating principle, characteristic parameters, and design guidelines of the system.
\end{abstract}

\section{Introduction}
Conventional levitation or hovering devices produce lift either by redirecting large air masses downward (drones, propellers) or by sealing a pressurized cushion with flexible skirts (hovercrafts).
The present concept introduces a different principle: the lift arises from a static overpressure trapped under a rigid disc, where the leakage is dynamically minimized by a thin annular high-speed jet forming a \emph{vertical air curtain}.
The curtain acts as a fluidic barrier, reducing communication between the inner cushion and the external environment. 
The resulting pressure field supports the load with minimal downward mass flux.

\begin{figure}[H]
  \centering
  \includegraphics[width=0.95\linewidth]{../figs/schema_geometry.png}
  \caption{Schematic of the aerodynamic levitation system: the outer annular air curtain seals the internal cushion; a central inlet compensates the leakage.}
  \label{fig:geometry}
\end{figure}

\section{System Description and Operating Principle}

\subsection{Geometry}
The system consists of a circular disc of outer radius $R$ hovering above a flat surface at a mean clearance $h$, with $h$ of the same order of magnitude as $R$. 
An annular slot of width $b_0$ is located near the periphery, at $r \approx R_o \simeq R$, and connected to a circular impeller or blower that feeds a downward high-speed jet of mean velocity $U_0$ and density $\rho$.
The jet forms a \emph{vertical curtain} along the disc edge, which impinges on the ground and partly turns radially outward. 
The region enclosed by the curtain ($r < R_i$) forms the \emph{cushion chamber}, whose mean pressure $p_c = p_\infty + \Delta p$ is greater than ambient.
A central inlet supplies a make-up flow $\dot{m}_{in}$ to compensate the residual losses $\dot{m}_{loss}$ leaking through the curtain and the side gap.
Figure~\ref{fig:geometry} illustrates the layout and main parameters.

\subsection{Operating Mechanism}
\begin{enumerate}
  \item The annular impeller accelerates air through the peripheral slot, creating a jet with local momentum per unit circumference
  \begin{equation}
      J = \rho U_0^2 b_0 \quad [\si{N/m}].
  \end{equation}
  This jet descends nearly vertically and interacts with the ground, generating a high-momentum barrier.
  \item The barrier strongly limits the outflow from the inner cushion. 
  The rate of leakage $\dot{m}_{loss}$ decreases with increasing $J$, since the curtain resists entrainment and backflow.
  \item The central inlet injects the make-up flow $\dot{m}_{in}$, which maintains the balance
  \begin{equation}
      \dot{m}_{in} = \dot{m}_{loss}(\Delta p,\,h,\,J),
  \end{equation}
  determining a steady cushion pressure $\Delta p$.
  \item The upward lift force is then
  \begin{equation}
      F_L = \Delta p\,A_i = \Delta p\,\pi R_i^2,
  \end{equation}
  which supports the disc and the carried load.
\end{enumerate}

\subsection{Governing Parameters}
Key relationships for estimation include:
\begin{itemize}
  \item Jet mass flow rate:
  \begin{equation}
      \dot{m}_{jet} = 2\pi R_o \rho U_0 b_0.
  \end{equation}
  \item Leakage flow (without curtain, approximated as a turbulent annular orifice):
  \begin{equation}
      \dot{m}_{leak,0} \approx 2\pi R \rho C_d h \sqrt{\frac{2\Delta p}{\rho}},
  \end{equation}
  where $C_d$ is a discharge coefficient.
  \item The curtain effectively reduces the leakage by a factor depending on the non-dimensional \emph{sealing number}
  \begin{equation}
      S = \frac{J}{\rho \Delta p h},
  \end{equation}
  which measures the ratio between the jet momentum and the pressure-driven leakage momentum. For $S \gg 1$, the seal is effective and $\dot{m}_{loss} \ll \dot{m}_{leak,0}$.
\end{itemize}

\subsection{Design Guidelines}
\begin{itemize}
  \item \textbf{Gap height $h$:} too large increases leakage; too small risks instability and contact. 
  Practical ratios are $h/R = 0.05$–$0.3$.
  \item \textbf{Slot width $b_0$:} kept small (millimetric scale) to maximize $U_0$ for a given jet mass flow.
  \item \textbf{Jet inclination:} a slight inward tilt (a few degrees) helps pressurize the inner region and reduce entrainment, avoiding a strong Coandă effect.
  \item \textbf{Pressure and lift:} the equilibrium overpressure scales with $\Delta p \sim J/(h\rho)$ for strong curtains, allowing significant lift with modest power.
  \item \textbf{Power balance:} the jet power $P_{jet}\sim \dot{m}_{jet} U_0^2/2$ and the make-up power $P_c\sim \dot{m}_{in}\Delta p/\rho$ determine the global efficiency.
  \item \textbf{Stability:} small perturbations in $h$ induce restoring pressure gradients (if one side approaches the ground, local $\Delta p$ rises), yielding passive self-leveling.
\end{itemize}

\subsection{Comparison with Existing Systems}
\begin{itemize}
  \item \textbf{Versus drones:} lift is not generated by downward acceleration of large air masses but by static overpressure.
  \item \textbf{Versus hovercrafts:} no flexible skirt or Coandă adhesion is used; the curtain provides a fluidic seal.
  \item \textbf{Advantages:} uniform lift distribution, reduced mechanical complexity, potentially quieter and more efficient operation at close ground distances.
\end{itemize}

\section{Conclusions}
The proposed aerodynamic levitator uses a jet-induced air curtain to dynamically seal a confined air cushion under a circular disc. 
The system decouples the functions of \emph{sealing} (provided by the annular jet) and \emph{pressurization} (from the central inlet), allowing the lift and stability to be controlled independently.
By tuning the jet momentum, gap height, and make-up flow, it is possible to sustain significant loads with modest flow rates.
Future work will focus on experimental validation, optimization of the curtain geometry, and analysis of stability and noise performance.

\end{document}
