\documentclass[11pt,a4paper]{article}
\usepackage[margin=2.5cm]{geometry}
\usepackage{amsmath, amssymb}
\usepackage{siunitx}
\usepackage{graphicx}
\usepackage{hyperref}

\title{Hover Disc a 20 cm: Modello 2D Assi-simmetrico e Sigillo Aerodinamico}
\author{}
\date{\today}

\begin{document}
\maketitle

\section{Premessa}
Consideriamo un disco di raggio $R$ sospeso ad una distanza $h_c \approx \SI{0.20}{m}$ dal suolo. La portanza è fornita da un cuscino d'aria con pressione quasi uniforme $p_c$ sull'area $\pi R^2$, mentre le perdite avvengono principalmente attraverso una regione anulare periferica di larghezza $w$ e \emph{gap efficace} $h_{\mathrm{eff}}$ determinato da una corona di getti (sigillo aerodinamico, effetto Coandă).

\section{Equazioni}
La pressione richiesta per sostenere un carico $m$ è
\begin{equation}
p_c = \frac{m g}{\pi R^2}.
\end{equation}

\subsection{Regime viscoso (lubrificazione)}
Nel tratto anulare $R-w \le r \le R$, assumendo flusso tra lastre parallele,
\begin{align}
\bar u(r) &= -\frac{h_{\mathrm{eff}}^2}{12\mu} \frac{\mathrm{d}p}{\mathrm{d}r},\\
Q &= 2\pi r h_{\mathrm{eff}} \bar u(r),
\end{align}
da cui
\begin{equation}
\frac{\mathrm{d}p}{\mathrm{d}r} = -\frac{6\mu Q}{\pi r h_{\mathrm{eff}}^3},
\end{equation}
e integrando
\begin{equation}
p_c = \frac{6\mu Q}{\pi h_{\mathrm{eff}}^3}\ln\!\left(\frac{R}{R-w}\right)
\quad\Rightarrow\quad
Q = \frac{\pi h_{\mathrm{eff}}^3 p_c}{6\mu \ln\!\left(\frac{R}{R-w}\right)}.
\end{equation}

\subsection{Regime orifizio (inerziale)}
Per numeri di Reynolds elevati il tratto anulare agisce come una luce di uscita con area $A_{\mathrm{eff}} = 2\pi R h_{\mathrm{eff}}$ e coefficiente di scarico $C_d$:
\begin{align}
Q &= C_d\, A_{\mathrm{eff}} \sqrt{\frac{2 p_c}{\rho}},\\
P_{\mathrm{ideal}} &= p_c\, Q.
\end{align}

\section{Campi $p(r)$ e $\bar u(r)$}
Nel regime viscoso si ottiene un decadimento logaritmico di pressione nel tratto anulare; nel regime orifizio la pressione è circa costante $p_c$ fino a $r \approx R$ con caduta netta all'uscita. Il codice produce i grafici e salva le figure in \texttt{figs/}.

\section{Uso}
\begin{verbatim}
python scripts/run_sim.py --R 1.0 --h_c 0.20 --w 0.05 --h_eff 0.03 --m_load 40 --cd 0.65
\end{verbatim}

\section{Note}
Il modello è stazionario, incomprimibile e 2D; il sigillo aerodinamico è rappresentato per mezzo di un gap efficace $h_{\mathrm{eff}}$ e/o di un coefficiente $C_d$. Estensioni possibili: doppia corona, anello di risucchio, suolo non piano, controllo attivo a settori.

\bibliographystyle{plain}
\bibliography{refs}
\end{document}
