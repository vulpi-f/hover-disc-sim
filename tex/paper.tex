\documentclass[11pt,a4paper]{article}
\usepackage[margin=2.5cm]{geometry}
\usepackage{amsmath, amssymb}
\usepackage{siunitx}
\usepackage{graphicx}
\usepackage{hyperref}

\title{Hover Disc a 20 cm: Modello 2D Assi-simmetrico e Sigillo Aerodinamico}
\author{}
\date{\today}

\begin{document}
\maketitle

\section{Premessa}
Consideriamo un disco di raggio \(R\) sospeso ad una distanza \(h_c \approx \SI{0.20}{m}\) dal suolo.  
La portanza è fornita da un cuscino d'aria con pressione quasi uniforme \(p_c\) sull'area \(\pi R^2\), mentre le perdite si concentrano in una regione anulare periferica di larghezza \(w\).  
Il sistema di sigillo aerodinamico (tramite corona di getti o effetto Coandă) introduce un \emph{gap efficace} \(h_{\mathrm{eff}}\) nella zona del sigillo.  

Lo script Python in questa repository implementa un modello 2D (assi-simmetrico) stazionario che calcola i campi di pressione e velocità (radiale e assiale) sulla sezione \(r\)–\(z\). In particolare, viene creato un plot di geometria e una mappa a colori con tre subplot: pressione \(p(r,z)\), velocità radiale \(u_r(r,z)\), velocità assiale \(u_z(r,z)\).

Qui di seguito sono elencate le principali assunzioni, le equazioni implementate e le modalità di costruzione dei campi.

\section{Assunzioni principali del modello}
\begin{itemize}
  \item Il flusso è **incomprimibile**, stazionario e assi-simmetrico.
  \item Nell'area interna \(0 \le r \le R - w\), si assume che la distanza dal suolo sia \(h_c\), la pressione sia circa uniforme \(p_c\) e le velocità trascurabili rispetto al flusso nel sigillo.
  \item Nel tratto anulare di sigillo \(R - w \le r \le R\), il flusso è modellato come flusso vischioso tra piastre parallele (applicazione dell'approssimazione di lubrificazione), con gap costante \(h_{\mathrm{eff}}\).
  \item Il modello non tiene conto dell'inerzia nel sigillo (non è un regime orifizio attivo con ingresso di getti distinti), per ottenere un campo regolare e continuo.
  \item Il campo assiale \(u_z\) nel sigillo è assunto ~0, coerente con il modello di flusso radiale puro in regime lubrificato (il flusso si “distribuisce” radialmente, senza penetrazione in z).
  \item All'esterno del disco (\(r > R\)) la pressione è posta uguale a quella ambiente (\(p=0\)) e le velocità sono nulle.
\end{itemize}

\section{Equazioni implementate}

\subsection{Pressione richiesta per la portanza}
Per sostenere un carico \(m\):
\[
p_c = \frac{m g}{\pi R^2}.
\]

\subsection{Portata \(Q\) nel tratto di sigillo — regime vischioso (lubrificazione)}
Nella zona sigillo \(R - w \le r \le R\), si assume:
\[
\bar u(r) = -\frac{h_{\mathrm{eff}}^2}{12\,\mu}\,\frac{dp}{dr}, \qquad
Q = 2\pi\,r\,h_{\mathrm{eff}}\,\bar u(r).
\]
Da qui:
\[
\frac{dp}{dr} = -\frac{6\,\mu\,Q}{\pi\,r\,h_{\mathrm{eff}}^3}.
\]
Integrando da \(r = R-w\) (dove \(p = p_c\)) fino a \(r = R\) (dove \(p = 0\)):
\[
p_c = \frac{6\,\mu\,Q}{\pi\,h_{\mathrm{eff}}^3} \ln\!\frac{R}{R-w}
\quad\Longrightarrow\quad
Q = \frac{\pi\,h_{\mathrm{eff}}^3\,p_c}{6\,\mu\,\ln\!\left(\tfrac{R}{R-w}\right)}.
\]
La potenza aerodinamica ideale è:
\[
P_{\mathrm{ideal}} = p_c\,Q.
\]

\subsection{Costruzione dei campi 2D}
Il codice costruisce una griglia \((r,z)\) con \(r\) da \(0\) a \(R + w\) e \(z\) da \(0\) a \(h_c\).  
Si definiscono tre regioni tramite maschere logiche:
\begin{itemize}
  \item \(\text{mask}_{\mathrm{interior}}\): \(r \le R - w\), \(z \le h_c\). In questa zona \(p = p_c\), \(u_r = 0\), \(u_z = 0\).
  \item \(\text{mask}_{\mathrm{seal}}\): \(R - w < r \le R\), \(z \le h_{\mathrm{eff}}\). In questa zona:
  \[
  p(r,z) = p_c - \underbrace{\frac{6\,\mu\,Q}{\pi\,h_{\mathrm{eff}}^3}}_{\text{costante}} \ln\!\frac{r}{R-w},
  \]
  \[
  u_r(r,z) = 6\,\bar u(r)\,\phi\,(1 - \phi), \quad \phi = \frac{z}{h_{\mathrm{eff}}}, \qquad u_z(r,z) \approx 0.
  \]
  \item \(\text{mask}_{\mathrm{outside}}\): \(r > R\). In questa zona \(p = 0\), \(u_r = u_z = 0\).
\end{itemize}

\noindent Dove \(\bar u(r)\) è la velocità media radiale:
\[
\bar u(r) = \frac{Q}{2\pi\,r\,h_{\mathrm{eff}}}.
\]

Con queste formulazioni, il codice produce:
\begin{itemize}
  \item il plot della geometria (sezione \(r\)–\(z\)) con indicazione del disco, labbro e zona di sigillo;
  \item il plot con tre subplot (colormap) per \(p(r,z)\), \(u_r(r,z)\), \(u_z(r,z)\).
\end{itemize}

\section{Risultati d'esempio}
Con i parametri tipici:
\[
R = 1.0\,\text{m}, \quad w = 0.05\,\text{m}, \quad h_{\mathrm{eff}} = 0.03\,\text{m}, \quad m = 40\,\mathrm{kg},
\]
il codice stampa i valori di \(p_c\), \(Q\) e \(P_{\mathrm{ideal}}\), e salva i file immagine in \texttt{figs/geometry.png} e \texttt{figs/fields\_tripanel.png}.

\section{Possibili estensioni}
\begin{itemize}
  \item Passare al regime orifizio (modello con \(\mathrm{Cd}\)) e generare un campo di pressione/velocità coerente con quel regime.
  \item Inserire un termine di alimentazione (o risucchio) che generi un componente \(u_z \neq 0\), più realistico per l'azione della corona.
  \item Modellare un doppio sigillo (corona interna + corona esterna) con slot di risucchio.
  \item Tenere conto di effetti di compressibilità, perdite locali, turbolenza e variazione del gap in funzione della quota.
  \item Estendere il modello a condizioni non piane del suolo o offset verticali variabili.
\end{itemize}

\end{document}
