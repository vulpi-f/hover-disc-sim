% !TEX program = pdflatex
\documentclass[11pt,a4paper]{article}

\usepackage[utf8]{inputenc}
\usepackage[T1]{fontenc}
\usepackage{lmodern}
\usepackage{geometry}
\geometry{margin=1in}
\usepackage{amsmath,amssymb}
\usepackage{siunitx}
\usepackage{graphicx}
\usepackage[font=small,labelfont=bf,labelsep=endash]{caption}
\usepackage{booktabs}
\usepackage{physics}
\usepackage{hyperref}
\hypersetup{colorlinks=true,linkcolor=blue,citecolor=blue,urlcolor=blue}

\title{Design Methodology for a Dual-Jet Hovering Disc with Concentric Air Curtains:\\
Model, Non-Dimensionalization, and Physical Assumptions}
\author{ }
\date{ }

\begin{document}
\maketitle

\begin{abstract}
This paper presents a design-oriented physical model for a hovering disc sustained by two concentric air jets: 
an outer annular jet that forms an aerodynamic curtain to confine the inner flow and an inner jet that maintains the central cushion pressure.
The analysis uses a low-Mach compressible, axisymmetric core model with an anisotropic Stokes--Darcy closure to approximate the pressure and velocity fields in the cushion region.
A detailed discussion of the model's assumptions and their physical validity is provided, along with a complete nomenclature of symbols used throughout the work.
\end{abstract}

\section{Geometry and Notation}
The geometry follows Fig.~\ref{fig:geometry}, defining the coordinate system and characteristic dimensions:
\begin{itemize}
  \item $R_{\mathrm{tot}}$ -- total radius of the disc.
  \item $h$ -- hovering height from the ground.
  \item $w$ -- width of the peripheral leakage ring; $R^{-}=R_{\mathrm{tot}}-w$ is its inner radius.
  \item $b$ -- thickness of the outer annular jet slot.
  \item $h_{\mathrm{eff}}$ -- effective sealing height (characteristic of curtain recirculation).
  \item $p_0$ -- ambient pressure; $p_c=W/(\pi R_{\mathrm{tot}}^2)$ -- cushion pressure supporting the load $W$.
  \item $U_{\mathrm{out}}$, $\rho_j$ -- speed and density of the outer jet.
  \item $\mu$ -- dynamic viscosity; $R_g$ -- specific gas constant for air.
  \item $\dot{m}_{\mathrm{in}}$ -- air mass flow entering internal region.
  \item $\dot{m}_{\mathrm{out}}$ -- air mass flow of outer region.
  \item $\dot{m}_{\mathrm{loss}}$ -- air mass flow exiting internal region.
\end{itemize}

\begin{figure}[t]
  \centering
  \includegraphics[width=0.95\linewidth]{../figs/schema_geometry.png}
  \caption{Schematic of the hovering disc with two concentric jets: the outer annular curtain and the central make-up flow.}
  \label{fig:geometry}
\end{figure}

\section{Model Overview}
The cushion region ($0\le r\le R^{-},\ 0\le z\le h$) is filled with air at variable pressure, temperature, and density. 
The mean flow satisfies an anisotropic Stokes--Darcy closure:
\begin{equation}
  u = -\frac{\kappa_r}{\mu}\,\partial_r p,\qquad
  w = -\frac{\kappa_z}{\mu}\,\partial_z p,
\end{equation}
with permeability coefficients $\kappa_r=\alpha_r h^2$ and $\kappa_z=\alpha_z h^2$, where $\alpha_r$ and $\alpha_z$ are empirical dimensionless parameters encoding the overall resistance of the confined air layer.

The continuity and state relations read
\begin{equation}
  \frac{1}{r}\,\partial_r\!\left(r\rho u\right)+\partial_z(\rho w)=0,\qquad
  p=\rho R_g T.
\end{equation}
Elimination of $u,w$ gives the pressure formulation:
\begin{equation}
  \frac{1}{r}\partial_r(r\rho\kappa_r\partial_r p)+\partial_z(\rho\kappa_z\partial_z p)=0.
\end{equation}

\section{Validity of Modeling Assumptions}
\subsection{Low-Mach Compressibility}
The jets have typical velocities $U_{\mathrm{out}}\approx30$--$60\,$m/s, leading to a Mach number $Ma=U/a\approx0.1$--$0.2$ with $a\simeq343\,$m/s. 
This regime justifies a \emph{low-Mach} formulation: the flow is compressible enough to exhibit pressure- and temperature-dependent density, but acoustic effects remain negligible. 
Hence, the ideal-gas relation $p=\rho R_g T$ is retained, while the flow is assumed quasi-static in time.

\subsection{Thermal Uniformity and Energy Exchange}
Although the confined air experiences some compression heating, the characteristic time scales of thermal diffusion and convective mixing by the curtain are short compared to global unsteadiness. 
The first-order model therefore assumes a uniform temperature $T=T_\infty$, with the understanding that future extensions may include the steady energy balance to recover small deviations of $T(r,z)$.

\subsection{Stokes--Darcy Closure}
The Stokes--Darcy model does not imply a porous medium in the literal sense. 
Instead, it approximates the momentum balance of a low-Reynolds, highly dissipative, confined flow.
At low $Re=\rho U h/\mu$, the Stokes equations reduce to a linear proportionality between pressure gradient and velocity.
Replacing $\nabla^2\mathbf{u}\sim \mathbf{u}/L^2$ with an effective geometric length $L\sim h$ yields
\begin{equation}
  \mathbf{u}\approx-\frac{h^2}{\mu}\nabla p,
\end{equation}
which is mathematically equivalent to Darcy’s law with an effective permeability $\kappa\sim h^2$.
The anisotropic form used here,
\begin{equation}
  \kappa_r=\alpha_r h^2,\qquad \kappa_z=\alpha_z h^2,
\end{equation}
accounts for different confinement levels in the radial and vertical directions.

\paragraph{Validity range.} 
This closure is valid provided that:
\begin{itemize}
  \item the Reynolds number in the cushion $Re_c=\rho U_c h/\mu \ll 1$;
  \item pressure variations are slow and inertia negligible;
  \item the flow is quasi-steady and dominated by viscous losses and boundary friction;
  \item local turbulence and recirculation effects are absorbed into the empirical coefficients $\alpha_r,\alpha_z$.
\end{itemize}
It is particularly suited to parametric design and control studies where the detailed jet micro-structure is not resolved.

\subsection{Boundary Conditions and Curtain Coupling}
The air curtain imposes a sealing pressure distribution along the rim $r=R^{-}$,
\begin{equation}
  p_{\mathrm{edge}}(z)=p_0+\Delta p\,\Phi(z/h),\qquad
  \Delta p=C_t\,\frac{\rho_j U_{\mathrm{out}}^2 b}{h_{\mathrm{eff}}}.
\end{equation}
Choosing $\Phi(z)$ increasing with $z$ ensures $\partial_z p>0$ and hence a downward core flow $w<0$ via the Darcy relation. 
On $z=0$ and $z=h$, the no-penetration condition ($\partial_z p=0$) holds in the reduced model; these planes are streamlines of the slow core flow.

\section{Non-Dimensionalization (Summary)}
Define dimensionless variables:
\begin{equation}
  \hat r=r/R_{\mathrm{tot}},\quad \hat z=z/h,\quad 
  \hat p=(p-p_0)/p_c,\quad \hat\rho=\rho/\rho_\infty,
\end{equation}
and permeability anisotropy $\mathcal{A}=(\alpha_z/\alpha_r)(R_{\mathrm{tot}}/h)^2$.
The governing equation becomes
\begin{equation}
  \frac{1}{\hat r}\partial_{\hat r}\!\left(\hat r\,\hat\rho\,\partial_{\hat r}\hat p\right)
  +\mathcal{A}\,\partial_{\hat z}\!\left(\hat\rho\,\partial_{\hat z}\hat p\right)=0,
\end{equation}
with boundary conditions
\begin{equation}
  \partial_{\hat r}\hat p(0,\hat z)=0,\qquad
  \partial_{\hat z}\hat p(\hat r,0)=\partial_{\hat z}\hat p(\hat r,1)=0,\qquad
  \hat p(\hat R^{-},\hat z)=\Pi_{\mathrm{edge}}\Phi(\hat z).
\end{equation}

\section{Nomenclature}
\begin{tabular}{@{}ll@{}}
\toprule
Symbol & Description \\ \midrule
$R_{\mathrm{tot}}$ & Total radius of the disc \\
$R^{-}$ & Inner radius of the leakage ring ($R^{-}=R_{\mathrm{tot}}-w$) \\
$w$ & Width of peripheral leakage region \\
$h$ & Hovering height (disc--ground gap) \\
$h_{\mathrm{eff}}$ & Effective sealing height at rim \\
$b$ & Slot thickness of the curtain jet \\
$U_{\mathrm{out}}$ & Outer jet velocity \\
$\rho_j$ & Density of outer jet \\
$\rho$ & Density in the core region \\
$p,p_0,p_c$ & Local, ambient, and cushion pressures \\
$T,T_\infty$ & Local and ambient temperatures \\
$\mu$ & Dynamic viscosity of air \\
$R_g$ & Specific gas constant of air \\
$W$ & Payload supported by cushion \\
$\kappa_r,\kappa_z$ & Effective permeabilities (radial/axial) \\
$\alpha_r,\alpha_z$ & Dimensionless permeability coefficients \\
$u,w$ & Velocity components (radial, vertical) \\
$\Phi(z)$ & Dimensionless rim-pressure distribution \\
$C_t$ & Curtain transfer coefficient \\
$\Delta p$ & Rim pressure increment \\
$\Pi_{\mathrm{edge}}$ & Dimensionless rim pressure amplitude \\
$\mathcal{A}$ & Permeability anisotropy parameter \\
$\hat r,\hat z,\hat p$ & Dimensionless coordinates and pressure \\
$\hat u,\hat w$ & Dimensionless velocity components \\
$S$ & Velocity anisotropy ratio $S=(\alpha_z/\alpha_r)(R_{\mathrm{tot}}/h)$ \\ \bottomrule
\end{tabular}

\end{document}
